%\documentclass[a4paper, review, endfloat, doubleblind]{elsarticle}
\documentclass[a4paper, review, endfloat]{elsarticle}

% Packages (optional)
\usepackage{amsmath}  % For mathematical symbols
\usepackage{graphicx} % For including images
\usepackage{hyperref} % For clickable links

% Title, author, and affiliation
\title{Digital Transformation in the Shipping Industry: a Network-Based Systematic Review}

\author[1]{Andreas Pittas}
\ead{20200114@stu.uol.ac.cy}

\author[1]{Yannes Filippopoulos}
\ead{y.filippopoulos@uol.ac.cy}

\author[2]{Zoran Lajic}
\ead{zlajic@angelicoussisgroup.com}

\author[1]{Luca Ferrarini\corref{cor1}}
\ead{luca@uol.ac.cy}

\cortext[cor1]{Corresponding author}

\affiliation[1]{organization={Department of Information Technologies, University of Limassol},
	city={Limassol},
	country={Cyprus}}
\affiliation[2]{organization={Department of Energy Efficiency, Angelicoussis Group},
	city={Athens},
	country={Greece}}

\begin{document}  % Begin the document
	
	\begin{abstract}
		The shipping industry is undergoing a profound digital transformation, driven by advancements in automation, artificial intelligence, blockchain, and the Internet of Things (IoT). These technologies enhance operational efficiency, optimize supply chain management, and improve sustainability by reducing emissions and fuel consumption. However, navigating this digital revolution requires a structured understanding of emerging trends, challenges, and opportunities.
		A network-based systematic review serves as a crucial methodological approach for researchers, enabling them to synthesize existing knowledge, identify research gaps, and develop informed strategies to leverage digital transformation effectively.
		By critically analyzing co-citation and co-authorship networks, modeling topics over time, and performing trend analysis, we gain insights on the current status of digital transformation within the shipping industry, ultimately guiding industry stakeholders and researchers ....XXXXXXXXXXXXXx
		Our results show that .....XXXXXXXXXXXXXXXXXXXXXX
	\end{abstract}
	
	\begin{keyword}
		digital transformation \sep shipping industry \sep systematic literature review \sep complex networks
	\end{keyword}
	
	\maketitle  % Generates the title, author, and abstract section
	
	\section{Introduction}
	XXXXXXXXXXXXXXXXXXXXXXXXXXXXXXXXXXX
	
	\section{Literature Review}
	
	\section{Methodology}
	You can describe your approach, methods, or framework here.
	
	\subsection{Keyword identification and data collection}
	We asked experts in the shipping industry to identify the most relevant keywords related to the industry itself and to digital technologies and digital transformation. Their analysis resulted in 35 keywords, listed in Table \ref{tab:keywords}.
	
	\begin{table}[h]
		\centering
		\caption{List of keywords identified by experts.}
		\begin{tabular}{l c}
			\hline
			Keyword& Type (Digit. Trans. or Shipping) \\
			\hline
			Digital transformation & Digit. Trans. \\
			Digital innovation & Digit. Trans. \\
			Digital ecosystems & Digit. Trans. \\
			Digitization & Digit. Trans. \\
			Digitalization & Digit. Trans. \\
			Digital platforms & Digit. Trans. \\
			Industry 4.0 & Digit. Trans. \\
			Smart technologies & Digit. Trans. \\
			Data-driven transformation & Digit. Trans. \\
			Automation & Digit. Trans. \\
			Internet of Things & Digit. Trans. \\
			Blockchain & Digit. Trans. \\
			Data analysis & Digit. Trans. \\
			Artificial intelligence & Digit. Trans. \\
			Machine learning & Digit. Trans. \\
			Big data & Digit. Trans. \\
			Cloud computing & Digit. Trans. \\
			Cyber-physical systems & Digit. Trans. \\
			Digital twins & Digit. Trans. \\
			Edge computing & Digit. Trans. \\
			5G networks & Digit. Trans. \\
			Predictive analytics & Digit. Trans. \\
			Cybersecurity & Digit. Trans. \\
			Supply chain integration & Digit. Trans. \\
			shipping & Shipping \\
			maritime & Shipping \\
			Sea freight & Shipping \\
			Smart ports & Shipping \\
			Autonomous ships & Shipping \\
			Fleet management & Shipping \\
			Cargo tracking & Shipping \\
			Digital shipyards & Shipping \\
			Port digitalization & Shipping \\
			Port automation & Shipping \\
			Vessel performance & Shipping \\
			\hline
		\end{tabular}
		\label{tab:keywords}
	\end{table}
	
	Data was collected from three research engines: EBSCO \citep{vaughan2011ebsco}, ProQuest \citep{cooke2017proquest}, and IEEE eXplore \citep{wilde2016ieee}. The search was performed on October the 22nd 2024. For each engine, we retrieved scientific articles containing any of the digital transformation related keywords and any of the shipping industry related keywords, in either their title or abstract. The exact query for each engine are available on request. We limited our results using the following criteria: \(a\) only English literature, and \(b\) only scientific contributions published in peer-reviewed journals. Table \ref{tab:searchres} shows the results.
	
	\begin{table}[h]
		\centering
		\caption{Number of retrieved articles per research engine.}
		\begin{tabular}{l c}
			\hline
			Engine & No. of scientific articles \\
			\hline
			EBSCO & 1904 \\
			ProQuest & 2011 \\
			IEEE eXplore & 300 \\
			\hline
		\end{tabular}
		\label{tab:searchres}
	\end{table}
	
	All search engines provided the digital object identifier for the articles. This allowed us to screen the resulting set and identify 2324 unique articles for the subsequent analysis. One challenge of using different data engines is the variety of attributes they return for each article. In order to have the same information for each article, we queried a fourth search engine for all the 2324 articles. We chose OpenAlex \citep{priem2022openalex}, which has been shown to be suitable for bibliometric analysis \citep{alperin2024analysis}. Our final result set comprised 2293 scientific publications.
	
	\subsection{Descriptive Statistics}
	We started our analysis evaluating descriptive statistics across our article set. More specifically, we calculated:
	\begin{enumerate}
		\item the distribution of the number of publications per year;
		\item the distribution of publications across authors, identifying the most prolific authors;
		\item the distribution of publications across institutions, identifying the research centers with the highest number of publications;
		\item the distribution of publications across countries.
	\end{enumerate}
	
	
	\section{Results}
	Present any results, tables, or figures.
	
	\section{Discussion}
	
	\section{Conclusion}
	Summarize key findings and future work.
	
	\bibliographystyle{elsarticle-harv}
	\bibliography{dxshippingbibliography}
	
\end{document}  % End of document